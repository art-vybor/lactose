% utf-8 ru, unix eolns
\documentclass[12pt,a4paper,oneside]{extarticle}
    \righthyphenmin=2 %минимально переносится 2 символа
    \sloppy

\usepackage{geometry} % А4, примерно 28-31 строк(а) на странице 
    \geometry{paper=a4paper}
    \geometry{includehead=false} % Нет верх. колонтитула
    \geometry{includefoot=true}  % Есть номер страницы
    \geometry{bindingoffset=0mm} % Переплет    : 0  мм
    \geometry{top=20mm}          % Поле верхнее: 20 мм
    \geometry{bottom=25mm}       % Поле нижнее : 25 мм 
    \geometry{left=30mm}         % Поле левое  : 30 мм (статья)
    \geometry{right=30mm}        % Поле правое : 30 мм (статья)
    \geometry{headsep=10mm}  % От края до верх. колонтитула: 10 мм
    \geometry{footskip=20mm} % От края до нижн. колонтитула: 20 мм 

\usepackage{cmap}
\usepackage[T2A]{fontenc} 
\usepackage[utf8x]{inputenc}
\usepackage[english,russian]{babel}
\usepackage{misccorr}

\usepackage{amsmath}
\usepackage{amsfonts}
\usepackage{amssymb}

\setlength{\parindent}{1.25cm}  % Абзацный отступ: 1,25 см
\usepackage{indentfirst}        % 1-й абзац имеет отступ

\usepackage{setspace}   

\onehalfspacing % Полуторный интервал между строками

\usepackage{clrscode}% псевдокод 

\makeatletter
\renewcommand{\@oddfoot }{\hfil\thepage\hfil} % Номер стр.
\renewcommand{\@evenfoot}{\hfil\thepage\hfil} % Номер стр.
\renewcommand{\@oddhead }{} % Нет верх. колонтитула
\renewcommand{\@evenhead}{} % Нет верх. колонтитула
\makeatother

\usepackage{fancyvrb}

\usepackage[pdftex]{graphicx}  % поддержка картинок 
   \usepackage{epstopdf}
   \DeclareGraphicsRule{.eps}{pdf}{.pdf}{`epstopdf #1}


\usepackage{syntax} %для поддержки рбнф
\setlength{\grammarindent}{12em} %устанавливает нужный отступ перед ::=
\setlength{\grammarparsep}{6pt plus 1pt minus 1pt}  %сокращает расстояние между правилами

\renewcommand{\labelenumi}{\theenumi.} %меняет вид нумерованного списка

\usepackage{perpage} %нумерация сносок 
\MakePerPage{footnote}

\usepackage[all]{xy} %поддержка графов

\usepackage{listings} %поддержка листингов
\lstset{language=Haskell, numbers=left}

%
% Обычный размер шрифта в заголовках
%
\usepackage[explicit]{titlesec} 
\titleformat{\section}{\normalfont\bfseries}{\thesection.}{0.5em}{#1}
\titleformat{\subsection}{\normalfont\bfseries}{\thesubsection.}{0.5em}{#1}
\titleformat{\subsubsection}{\normalfont\bfseries}{\thesubsection.}{0.5em}{#1}


\begin{document}
\noindent УДК ...
\bigskip

\noindent\textbf{Препроцессор синаксического сахара для языка Scheme}

\noindent МГТУ им. Н.Э.Баумана\\
Выборнов А.И., Дубанов А.В.\\
art-vybor@ya.ru; qrcs@mail.ru

\noindent\textbf{Аннотация.} В работе рассмотрены создание препроцессора для языка Scheme и разработка входного языка для препроцессора. Препроцессор реализован в виде приложения с консольным интерфейсом. Исходные тексты проекта доступны по адресу: https://github.com/art-vybor/lactose

\noindent\textbf{Ключевые слова:} Lisp, Scheme, Синтаксический сахар, Компилятор, Препроцессор.

\bigskip

\noindent\textbf{Введение.}
Язык Lisp создавался в ходе работ по созданию искусственного интеллекта, но сейчас сфера его применения выходит далеко за пределы этой узкой области. Lisp это функциональный язык с динамической системой типов.
Его отличительной особенностью является то, что программы и данные являются списками символов.
Программа на языке Lisp представляет собой уже синтаксическое дерево, что облегчает создание парсеров этого языка.

В настоящее время Lisp представляет собой целое семейство языков программирования.
В данной работе рассматривается Scheme~---~один из наиболее известных диалектов Lisp.
В Scheme упор был сделан на простоту языка~---~в нём содержиться минимальное количество примитивных конструкций, которые, тем не менее, позволяют выразить любой требуемый функционал.

Несмотря на простоту языка Scheme, он перегружен большим количеством скобок.
Также Scheme не поддерживает инфиксную запись выражений, что, к примеру, не позволяет записывать арифметические выражения в привычном виде.
С другой стороны такие современные языки программирования как Haskell, Python или Erlang, позволяют писать более удобочитаемый код.

В рамках данной работы была сделана попытка освежить Sсheme, добавить ему элегантности, сделать код более удобочитаемым.
Для этого был разработан входной язык, обладающий синтаксическим сахаром, позаимствованным из Haskell и Python.
С помощью написанного препроцессора, входной язык комплируется в Sсheme.

\noindent\textbf{Входной язык препроцессора.}
Программа на входном языке представляет собой множество функций. Переменные также являются функциями. Если в корне программы присутствует объявление функции main~---~то эта функция будет интерпретирована как точка входа. Можно определять функции на Scheme. Токены языка представляют собой токены на языке Sсheme.

Рассмотрим более подробное описание входного языка:
    
    символы как в лиспе, за исключением... , так как инфиксная арифметика нужна. Пример:
    \begin{lstlisting}[mathescape] 
#t 123 #b010  "abc"  #\a  
    \end{lstlisting}

    инфиксные выражения. условия, вызовы функций, объявления функций тоже являются выражениями. приоритеты операций. Пример:
    \begin{lstlisting}[mathescape] 
2+2*2
(2+2)*2
    \end{lstlisting}
    

    условия Пример:
    \begin{lstlisting}[mathescape] 
if a then b else c
    \end{lstlisting}
     
    объявление функции - набор выражений разделённых ; последнее возвращается. Ключевое слово def. Пример:
    \begin{lstlisting}[mathescape] 
def fact n =
    def loop i = if i < n then i*(loop i+1) else n;
    loop 1
    \end{lstlisting}

    вызов функции. Пример:
    \begin{lstlisting}[mathescape] 
g (f a b)
    \end{lstlisting}

    вставка кода scheme 
    \begin{lstlisting}[mathescape] 
        { ... } export a,b,c
    \end{lstlisting}

    проблемы, недостатки

\noindent\textbf{Реализация препроцессора}
        общая структура: входной файл -> antlr syntax tree -> AST -> lisp_tree -> lisp_file

        особоенности каждого этапа преобразования

        Обработка ошибок

        Видимость символов
        
    Интерфейс
        CLI все дела

        примеры использования

\noindent\textbf{Cравнение входного языка с альтернативами}
входной язык; полученный lisp; lisp; python; haskell

\noindent\textbf{Заключение}
    Что получилось. Привнесена няшность, но возможности далеко не такие же как были.

\begin{thebibliography}{00}
\addcontentsline{toc}{section}{Литература}
    \bibitem{r5rs} стандарт r5rs
    \bibitem{dragon} axо ульман книга дракона
    \bibitem{antlrs} какая нибудь дока по antlr
\end{thebibliography}

%     \bibitem{mapreduce}
%         Jeffrey Dean, Sanjay Ghemawat. MapReduce: Simplified Data Processing on Large Clusters~//~Google, Inc.~---~2004.

%     \bibitem{problem}
%         MapReduce~//~slideshare: URL: \newline
%         http://www.slideshare.net/yandex/mapreduce-12321523

%     \bibitem{serialize}
%         What is the most efficient way to serialize in Python?~//~Quora: URL:  \newline
%         http://www.quora.com/What-is-the-most-efficient-way-to-serialize-in-Python

%     \bibitem{mr_tasks}
%         Hadoop save the World?~//~CODE1NSTINCT: URL:  \newline
%         http://www.codeinstinct.pro/2012/08/hadoop-design.html

%     \bibitem{mr_patterns}
%         MapReduce Patterns, Algorithms, and Use Cases~//~Highly Scalable Blog: URL:  \newline
%         https://highlyscalable.wordpress.com/2012/02/01/mapreduce-patterns/
        
% \end{thebibliography}

\end{document}