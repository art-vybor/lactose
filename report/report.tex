% utf-8 ru, unix eolns
\documentclass[12pt,a4paper,oneside]{extarticle}
    \righthyphenmin=2 %минимально переносится 2 символа %%%
    \sloppy

% Рукопись оформлена в соответствии с правилами оформления 
% электронной версии авторского оригинала, 
% принятыми в Издательстве МГТУ им. Н.Э. Баумана.

\usepackage{geometry} % А4, примерно 28-31 строк(а) на странице 
    \geometry{paper=a4paper}
    \geometry{includehead=false} % Нет верх. колонтитула
    \geometry{includefoot=true}  % Есть номер страницы
    \geometry{bindingoffset=0mm} % Переплет    : 0  мм
    \geometry{top=20mm}          % Поле верхнее: 20 мм
    \geometry{bottom=25mm}       % Поле нижнее : 25 мм 
    \geometry{left=25mm}         % Поле левое  : 25 мм
    \geometry{right=25mm}        % Поле правое : 25 мм
    \geometry{headsep=10mm}  % От края до верх. колонтитула: 10 мм
    \geometry{footskip=20mm} % От края до нижн. колонтитула: 20 мм 

\usepackage{cmap}
\usepackage[T2A]{fontenc} 
\usepackage[utf8x]{inputenc}
\usepackage[english,russian]{babel}
\usepackage{misccorr}

\usepackage{amsmath}
\usepackage{amsfonts}
\usepackage{amssymb}

%\usepackage{cm-super} %человеческий рендер русских шрифтов

\setlength{\parindent}{1.25cm}  % Абзацный отступ: 1,25 см
\usepackage{indentfirst}        % 1-й абзац имеет отступ

\usepackage{setspace}   

\onehalfspacing % Полуторный интервал между строками

\makeatletter
\renewcommand{\@oddfoot }{\hfil\thepage\hfil} % Номер стр.
\renewcommand{\@evenfoot}{\hfil\thepage\hfil} % Номер стр.
\renewcommand{\@oddhead }{} % Нет верх. колонтитула
\renewcommand{\@evenhead}{} % Нет верх. колонтитула
\makeatother

\usepackage{fancyvrb}


\usepackage[pdftex]{graphicx}  % поддержка картинок для пдф
\graphicspath{ {./pictures/} }
\usepackage{rotating}
%\DeclareGraphicsExtensions{.jpg,.png}




\renewcommand{\labelenumi}{\theenumi.} %меняет вид нумерованного списка

\usepackage{perpage} %нумерация сносок 
\MakePerPage{footnote}

\usepackage[all]{xy} %поддержка графов

\usepackage{listings} %листинги


\usepackage{url}


\usepackage{tikz} %для рисования графиков
\usepackage{pgfplots}


\usepackage{ccaption}%изменяет подпись к рисунку
\makeatletter 
\renewcommand{\fnum@figure}[1]{Рисунок~\thefigure~---~\sffamily}
\makeatother

\begin{document}
\pgfplotsset{compat=1.8}

\thispagestyle{empty}
\newpage
{
\centering


\textbf{
МОСКОВСКИЙ ГОСУДАРСТВЕННЫЙ ТЕХНИЧЕСКИЙ УНИВЕРСИТЕТ ИМЕНИ Н. Э. БАУМАНА \\
Факультет информатики и систем управления \\
Кафедра теоретической информатики и компьютерных технологий}
\bigskip
\bigskip
\bigskip
\bigskip
\bigskip
\bigskip
\bigskip

\vfill


Курсовой проект \\
по курсу <<Компиляторы>>

\bigskip

{\large <<Syntax sugar for Scheme>>}
\bigskip

\vfill



\hfill\parbox{4cm} {
Выполнил:\\
студент ИУ9-101 \hfill \\
Выборнов А. И.\hfill \medskip\\
Руководитель:\\
Дубанов А. В.\hfill
}


\vspace{\fill}

Москва \number\year
\clearpage
}


\tableofcontents

\clearpage


\section*{Введение}
\addcontentsline{toc}{section}{Введение}
    В двух словах о крутости лиспа и избыточности скобочек
\clearpage

\section{Теоретическая часть}
    \subsection{Lisp}
        Поподробнее про лисп.
    \subsection{Функционал входного языка}
        вступленние
        программа представляет собой список из определений функций и их вызовов

        мб ввести main и оставить только определение функций?

        \subsubsection{символы}
            как в лиспе, за исключением... 
            примеры
        \subsubsection{функции}
            1)sign x  | x > 0  = 1 (multiwayif)

                | x == 0 = 0

                | x < 0  = -1
            2) sign 0 = 0 
            sign a = if a > 0 then 1 else -1

            пример на Algebraic data type ?! я не работаю с типами :(

            3) \\x y -> x+y 

            какой должен быть вызов функции? мб сделать lisp-style?
        \subsubsection{условия}
            if a then b else c

        \subsubsection{variable bindings}
            where ...
        \subsubsection{инфиксная арифметика}
            при чём тут guile?
        \subsubsection{определение функций на scheme}
            мб вида: sign x: scheme (scheme body) ?

        
\clearpage

\section{Объекты и методы}
\label{sec:configuration} 
        \noindent Характеристики программного обеспечения:
        \begin{itemize}
            \item Операционная система --- ОpenSUSE 12.2 x86\textunderscore 64.
            \item Язык программирования --- Python 2.7.3.
        \end{itemize}
        
        \noindent Характеристики оборудования:
        \begin{itemize}
            \item Процессор --- Intel Core 2 Duo E6550 2.33 Гц 2 ядра.
            \item Оперативная память --- 2 Гбайт DDR2.
        \end{itemize}
\clearpage

\section{Реализация}
    \subsection{Используемые технологии}
        \subsubsection{ANTLR}
            что круто, а что не очень
            обязательно про минусы python версии
        \subsubsection{graphviz}
            ...
        \subsubsection{тестирование}
            ...
        \subsubsection{сборка пакета}
            ...

    \subsubsection{Детали реализации}
        общая структура:

        $входной файл -> antlr syntax tree -> AST -> lisp_tree -> lisp_file$

        особоенности каждого этапа преобразования
        \subsubsection{Обработка ошибок}
            сделано
        \subsubsection{Видимость символов}
            увы нет, надо обязательно доделать
        
    \subsection{Интерфейс}
        CLI все дела

        примеры использования

        

\clearpage

\section{Тестирование}
хз    
\clearpage

\section{Заключение}
    Что получилось. Привнесена няшность, но возможности далеко не такие как были.
\clearpage


\begin{thebibliography}{0}
\addcontentsline{toc}{section}{Список литературы}
    \bibitem{r5rs} стандарт r5rs
    \bibitem{dragon} axо ульман книга дракона
    \bibitem{antlrs} какая нибудь дока по antlr

    \bibitem{mapreduce}
        Jeffrey Dean, Sanjay Ghemawat. MapReduce: Simplified Data Processing on Large Clusters~//~Google, Inc.~---~2004.

    \bibitem{problem}
        MapReduce~//~slideshare: URL: \newline
        http://www.slideshare.net/yandex/mapreduce-12321523

    \bibitem{serialize}
        What is the most efficient way to serialize in Python?~//~Quora: URL:  \newline
        http://www.quora.com/What-is-the-most-efficient-way-to-serialize-in-Python

    \bibitem{mr_tasks}
        Hadoop save the World?~//~CODE1NSTINCT: URL:  \newline
        http://www.codeinstinct.pro/2012/08/hadoop-design.html

    \bibitem{mr_patterns}
        MapReduce Patterns, Algorithms, and Use Cases~//~Highly Scalable Blog: URL:  \newline
        https://highlyscalable.wordpress.com/2012/02/01/mapreduce-patterns/
        
\end{thebibliography}

\end{document}